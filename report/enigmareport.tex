%%%%%%%%%%%%%%%%%%%%%%%%%%%%%%%%%%%%%%%%%
% University Assignment Title Page 
% LaTeX Template
% Version 1.0 (27/12/12)
%
% This template has been downloaded from:
% http://www.LaTeXTemplates.com
%
% Original author:
% WikiBooks (http://en.wikibooks.org/wiki/LaTeX/Title_Creation)
%
% License:
% CC BY-NC-SA 3.0 (http://creativecommons.org/licenses/by-nc-sa/3.0/)
% 
% Instructions for using this template:
% This title page is capable of being compiled as is. This is not useful for 
% including it in another document. To do this, you have two options: 
%
% 1) Copy/paste everything between \begin{document} and \end{document} 
% starting at \begin{titlepage} and paste this into another LaTeX file where you 
% want your title page.
% OR
% 2) Remove everything outside the \begin{titlepage} and \end{titlepage} and 
% move this file to the same directory as the LaTeX file you wish to add it to. 
% Then add \input{./title_page_1.tex} to your LaTeX file where you want your
% title page.
%
%%%%%%%%%%%%%%%%%%%%%%%%%%%%%%%%%%%%%%%%%
%\title{Title page with logo}
%----------------------------------------------------------------------------------------
%	PACKAGES AND OTHER DOCUMENT CONFIGURATIONS
%----------------------------------------------------------------------------------------

\documentclass[12pt]{article}
\usepackage[french]{babel}
\usepackage[utf8]{inputenc}
\usepackage[T1]{fontenc}
%\usepackage{bookman} % or 'times'
\usepackage[top=3cm, bottom=3cm, left=3cm, right=3cm]{geometry}
\usepackage{fancyhdr}
\usepackage{lastpage}
\usepackage{graphicx}
\usepackage[colorinlistoftodos]{todonotes}
%\usepackage{hyperref}
%\usepackage{url}
\usepackage{pdfpages}
\usepackage{tocloft}

\begin{document}
\begin{titlepage}


\newcommand{\imagepath}{../image}
\newcommand{\university}{Université Libre de Bruxelles}
\newcommand{\faculty}{Département d'informatique}
\newcommand{\course}{INFO-F-308 Projet d'informatique 3 transdisciplinaire}
\newcommand{\authors}{Hakim \textsc{Boulahya}}
\newcommand{\supervisor}{Olivier \textsc{Markowitch}}
\newcommand{\HRule}{\rule{\linewidth}{0.5mm}} % Defines a new command for the horizontal lines, change thickness here

\center % Center everything on the page
 
%----------------------------------------------------------------------------------------
%	HEADING SECTIONS
%----------------------------------------------------------------------------------------

\textsc{\LARGE \university}\\[1.5cm] % Name of your university/college
\textsc{\Large \faculty}\\[0.5cm] % Major heading such as course name
\textsc{\large \course}\\[0.5cm] % Minor heading such as course title

%----------------------------------------------------------------------------------------
%	TITLE SECTION
%----------------------------------------------------------------------------------------

\HRule \\[0.4cm]
{ \huge \bfseries Simulateur graphique de la machine Enigma I}\\[0.4cm] % Title of your document
\HRule \\[1.5cm]
 
%----------------------------------------------------------------------------------------
%	AUTHOR SECTION
%----------------------------------------------------------------------------------------

\begin{minipage}{0.4\textwidth}
\begin{flushleft} \large
\emph{Auteur:}\\
\authors
\end{flushleft}
\end{minipage}
~
\begin{minipage}{0.4\textwidth}
\begin{flushright} \large
\emph{Superviseur:} \\
\supervisor % Supervisor's Name
\end{flushright}
\end{minipage}\\[2cm]

% If you don't want a supervisor, uncomment the two lines below and remove the section above
%\Large \emph{Author:}\\
%John \textsc{Smith}\\[3cm] % Your name

%----------------------------------------------------------------------------------------
%	DATE SECTION
%----------------------------------------------------------------------------------------

{\large Année académique 2015-2016}\\[2cm] % Date, change the \today to a set date if you want to be precise

%----------------------------------------------------------------------------------------
%	LOGO SECTION
%----------------------------------------------------------------------------------------

\includegraphics{\imagepath/logoULB.png}\\[1cm] % Include a department/university logo - this will require the graphicx package
 
%----------------------------------------------------------------------------------------

\vfill % Fill the rest of the page with whitespace

\end{titlepage}

\renewcommand{\cftsecleader}{\cftdotfill{\cftdotsep}}
\tableofcontents
\newpage

\pagestyle{fancy}

\setlength\headheight{15pt}
\renewcommand{\sectionmark}[1]{\markright{ #1}}
\renewcommand{\headrulewidth}{0.5pt}
\pagestyle{fancy}
\fancyhead[C]{} 
\fancyhead[RO]{\rightmark}
\fancyhead[RO]{\textsc{H. Boulahya}}
\fancyfoot[L]{Année académique 2015-2016}
\fancyfoot[R]{Page \thepage\ | \pageref{LastPage}}
\fancyfoot[C]{}

\begin{abstract}
Your abstract.
\end{abstract}

\section{Introduction}

%\url{http://i.telegraph.co.uk/multimedia/archive/03370/doge_3370416k.jpg}

%\label{sec:examples}

%\subsection{Sections}

%Use section and subsection commands to organize your document. \LaTeX{} handles all the formatting and numbering automatically. Use ref and label commands for cross-references.

% \section{Comments}

% Comments can be added to the margins of the document using the \todo{Here's a comment in the margin!} todo command, as shown in the example on the right. You can also add inline comments too:

% \todo[inline, color=green!40]{This is an inline comment.}

\begin{thebibliography}{4}
\bibitem{mainarticle} 
Authors,
\textit{``Title``},
Publisher,
accessed October -, 2015, doi: 
 
\end{thebibliography}

\newpage
\appendix




\end{document}
