\documentclass[letterpaper]{article}
\usepackage{natbib,alifexi}

\title{Simulation graphique de la machine Enigma I}
\author{Hakim Boulahya \\
\mbox{}\\
Université Libre de Bruxelles, Belgique\\
hboulahy@ulb.ac.be}


\begin{document}
\maketitle

\begin{abstract}
\end{abstract}

\section{Introduction}
Dans le cadre d'un projet transdiscplinaire, il m'a été demandé d'implémenter
un simulateur graphique de la machine Enigma I. Cette machine eletro-mécanique a été utilisé par
l'armée Nazi durant la deuxième guerre mondiale. Elle a permis aux militaires Allemands
de maintenir une commnunication secrète durant une majeur partie de la guerre. Très peu
de machine ont survécu du au fait que les Alliés ont recu l'ordre de detruire toute machine
enigma trouvé.

\section{La machine}

\subsection{composition}

nombre de possibilté

cracké par Alan Turring donc pas secu

\section{Etat de l'art}
\subsection{Util}
\subsection{Graphic}

\section{Mon simulateur}

\subsection{util}
\subsection{graphic}

\section{conclusion}

\footnotesize
\bibliographystyle{apalike}
\bibliography{example}


\end{document}
